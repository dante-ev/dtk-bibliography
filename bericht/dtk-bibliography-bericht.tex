% Title:       Frühjahrsputz im Archiv der DTK
% Author:      Leo Arnold (@leoarnold), Uwe Ziegenhagen (@UweZiegenhagen)
% URL:         https://github.com/dante-ev/dtk-bibliography
% License:     CC BY 4.0 - http://creativecommons.org/licenses/by/4.0/

\documentclass[ngerman]{dtk} 

\ifluatex
  \usepackage[utf8]{luainputenc}
\else
  \usepackage[utf8]{inputenc}
\fi

\usepackage{listings}

\title{Frühjahrsputz im Archiv der DTK} 
\Author{Leo}{Arnold}{Garching b. München}
\Author{Uwe}{Ziegenhagen}{Köln}

\addbibresource{../dtk-bibliography-ascii.bib}

\begin{document}
\maketitle

\begin{abstract}
Die alte Bibliographie der \TeX{}nischen Komödie wurde durch nicht-optimale Skripte unbrauchbar und geriet fast in Vergessenheit. Nach einer \enquote{Tour~de~Force} durch über hundert Ausgaben gibt es nun endlich wieder ein einheitliches Verzeichnis aller jemals in der \TeX nischen Komödie publizierten Beträge.
\end{abstract} 

\section{Ausgangspunkt}

Ausgangspunkt für dieses Projekt war die Frage, wieviele Artikel ich (Uwe) denn in den letzten Jahren in der DTK veröffentlicht hatte. Ein manuelles Durchschauen aller Ausgaben war zu aufwändig, daher habe ich bei der Redaktion der DTK nachgefragt. Herbert Voß war auch so freundlich, nmir die Gesamtbibliografie der DTK zuzusenden, verbunden mit dem Hinweis, dass die letze Aktualisierung schon einige Zeit hersei und es mit den Skripten, die die Bibliografie aus den DTK-Ausgaben extrahieren würden, einige Probleme gäbe.

Bei näherer Untersuchung stellte sich das Problem dann leider etwas größer dar als gedacht: Umlaute waren in allen möglichen Encodings und Schreibweisen vorhanden, Zeitangaben oft variabel gestaltet, wo keine Variablilität gewünscht ist, kurz: Es war ein ziemliches Durcheinander!

Auf der Froscon in St. Augustin schließlich gelang es, Leo Arnold aus München für dieses Projekt zu begeistern, der auch die Hauptarbeit bei der Reiningung auf sich lud und dem daher hier großer Dank zu zollen ist. 

\section{Der Reinigungsprozess}

@Leo: Du kannst das sicher besser beschreiben.


\section{Von Spinnweben bedeckt}

\nocite{*}

\begin{lstlisting}
Hallo
\end{lstlisting}


\section{Fazit}

Mit viel Aufwand haben wir jetzt wieder eine einheitliche Bibliografie, die alle jemals in der DTK erschienenen Artikel auflistet. Das Projekt findet sich auf github unter <URL>, wir werden es auch entsprechend von der Vereinswebseite verlinken.

\printbibliography

\end{document}